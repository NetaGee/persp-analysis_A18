\documentclass[]{article}
\usepackage{lmodern}
\usepackage{amssymb,amsmath}
\usepackage{ifxetex,ifluatex}
\usepackage{fixltx2e} % provides \textsubscript
\ifnum 0\ifxetex 1\fi\ifluatex 1\fi=0 % if pdftex
  \usepackage[T1]{fontenc}
  \usepackage[utf8]{inputenc}
\else % if luatex or xelatex
  \ifxetex
    \usepackage{mathspec}
  \else
    \usepackage{fontspec}
  \fi
  \defaultfontfeatures{Ligatures=TeX,Scale=MatchLowercase}
\fi
% use upquote if available, for straight quotes in verbatim environments
\IfFileExists{upquote.sty}{\usepackage{upquote}}{}
% use microtype if available
\IfFileExists{microtype.sty}{%
\usepackage{microtype}
\UseMicrotypeSet[protrusion]{basicmath} % disable protrusion for tt fonts
}{}
\usepackage[margin=1in]{geometry}
\usepackage{hyperref}
\hypersetup{unicode=true,
            pdftitle={MACS PS1},
            pdfauthor={Neta Grossfeld},
            pdfborder={0 0 0},
            breaklinks=true}
\urlstyle{same}  % don't use monospace font for urls
\usepackage{color}
\usepackage{fancyvrb}
\newcommand{\VerbBar}{|}
\newcommand{\VERB}{\Verb[commandchars=\\\{\}]}
\DefineVerbatimEnvironment{Highlighting}{Verbatim}{commandchars=\\\{\}}
% Add ',fontsize=\small' for more characters per line
\usepackage{framed}
\definecolor{shadecolor}{RGB}{248,248,248}
\newenvironment{Shaded}{\begin{snugshade}}{\end{snugshade}}
\newcommand{\KeywordTok}[1]{\textcolor[rgb]{0.13,0.29,0.53}{\textbf{{#1}}}}
\newcommand{\DataTypeTok}[1]{\textcolor[rgb]{0.13,0.29,0.53}{{#1}}}
\newcommand{\DecValTok}[1]{\textcolor[rgb]{0.00,0.00,0.81}{{#1}}}
\newcommand{\BaseNTok}[1]{\textcolor[rgb]{0.00,0.00,0.81}{{#1}}}
\newcommand{\FloatTok}[1]{\textcolor[rgb]{0.00,0.00,0.81}{{#1}}}
\newcommand{\ConstantTok}[1]{\textcolor[rgb]{0.00,0.00,0.00}{{#1}}}
\newcommand{\CharTok}[1]{\textcolor[rgb]{0.31,0.60,0.02}{{#1}}}
\newcommand{\SpecialCharTok}[1]{\textcolor[rgb]{0.00,0.00,0.00}{{#1}}}
\newcommand{\StringTok}[1]{\textcolor[rgb]{0.31,0.60,0.02}{{#1}}}
\newcommand{\VerbatimStringTok}[1]{\textcolor[rgb]{0.31,0.60,0.02}{{#1}}}
\newcommand{\SpecialStringTok}[1]{\textcolor[rgb]{0.31,0.60,0.02}{{#1}}}
\newcommand{\ImportTok}[1]{{#1}}
\newcommand{\CommentTok}[1]{\textcolor[rgb]{0.56,0.35,0.01}{\textit{{#1}}}}
\newcommand{\DocumentationTok}[1]{\textcolor[rgb]{0.56,0.35,0.01}{\textbf{\textit{{#1}}}}}
\newcommand{\AnnotationTok}[1]{\textcolor[rgb]{0.56,0.35,0.01}{\textbf{\textit{{#1}}}}}
\newcommand{\CommentVarTok}[1]{\textcolor[rgb]{0.56,0.35,0.01}{\textbf{\textit{{#1}}}}}
\newcommand{\OtherTok}[1]{\textcolor[rgb]{0.56,0.35,0.01}{{#1}}}
\newcommand{\FunctionTok}[1]{\textcolor[rgb]{0.00,0.00,0.00}{{#1}}}
\newcommand{\VariableTok}[1]{\textcolor[rgb]{0.00,0.00,0.00}{{#1}}}
\newcommand{\ControlFlowTok}[1]{\textcolor[rgb]{0.13,0.29,0.53}{\textbf{{#1}}}}
\newcommand{\OperatorTok}[1]{\textcolor[rgb]{0.81,0.36,0.00}{\textbf{{#1}}}}
\newcommand{\BuiltInTok}[1]{{#1}}
\newcommand{\ExtensionTok}[1]{{#1}}
\newcommand{\PreprocessorTok}[1]{\textcolor[rgb]{0.56,0.35,0.01}{\textit{{#1}}}}
\newcommand{\AttributeTok}[1]{\textcolor[rgb]{0.77,0.63,0.00}{{#1}}}
\newcommand{\RegionMarkerTok}[1]{{#1}}
\newcommand{\InformationTok}[1]{\textcolor[rgb]{0.56,0.35,0.01}{\textbf{\textit{{#1}}}}}
\newcommand{\WarningTok}[1]{\textcolor[rgb]{0.56,0.35,0.01}{\textbf{\textit{{#1}}}}}
\newcommand{\AlertTok}[1]{\textcolor[rgb]{0.94,0.16,0.16}{{#1}}}
\newcommand{\ErrorTok}[1]{\textcolor[rgb]{0.64,0.00,0.00}{\textbf{{#1}}}}
\newcommand{\NormalTok}[1]{{#1}}
\usepackage{graphicx,grffile}
\makeatletter
\def\maxwidth{\ifdim\Gin@nat@width>\linewidth\linewidth\else\Gin@nat@width\fi}
\def\maxheight{\ifdim\Gin@nat@height>\textheight\textheight\else\Gin@nat@height\fi}
\makeatother
% Scale images if necessary, so that they will not overflow the page
% margins by default, and it is still possible to overwrite the defaults
% using explicit options in \includegraphics[width, height, ...]{}
\setkeys{Gin}{width=\maxwidth,height=\maxheight,keepaspectratio}
\IfFileExists{parskip.sty}{%
\usepackage{parskip}
}{% else
\setlength{\parindent}{0pt}
\setlength{\parskip}{6pt plus 2pt minus 1pt}
}
\setlength{\emergencystretch}{3em}  % prevent overfull lines
\providecommand{\tightlist}{%
  \setlength{\itemsep}{0pt}\setlength{\parskip}{0pt}}
\setcounter{secnumdepth}{0}
% Redefines (sub)paragraphs to behave more like sections
\ifx\paragraph\undefined\else
\let\oldparagraph\paragraph
\renewcommand{\paragraph}[1]{\oldparagraph{#1}\mbox{}}
\fi
\ifx\subparagraph\undefined\else
\let\oldsubparagraph\subparagraph
\renewcommand{\subparagraph}[1]{\oldsubparagraph{#1}\mbox{}}
\fi

%%% Use protect on footnotes to avoid problems with footnotes in titles
\let\rmarkdownfootnote\footnote%
\def\footnote{\protect\rmarkdownfootnote}

%%% Change title format to be more compact
\usepackage{titling}

% Create subtitle command for use in maketitle
\newcommand{\subtitle}[1]{
  \posttitle{
    \begin{center}\large#1\end{center}
    }
}

\setlength{\droptitle}{-2em}

  \title{MACS PS1}
    \pretitle{\vspace{\droptitle}\centering\huge}
  \posttitle{\par}
    \author{Neta Grossfeld}
    \preauthor{\centering\large\emph}
  \postauthor{\par}
      \predate{\centering\large\emph}
  \postdate{\par}
    \date{10/14/2018}


\begin{document}
\maketitle

\begin{Shaded}
\begin{Highlighting}[]
\KeywordTok{getwd}\NormalTok{()}
\end{Highlighting}
\end{Shaded}

\begin{verbatim}
## [1] "/Users/netagrossfeld/Desktop/persp-analysis_A18/Assignments/A2"
\end{verbatim}

\begin{Shaded}
\begin{Highlighting}[]
\KeywordTok{setwd}\NormalTok{(}\StringTok{"/Users/netagrossfeld/Desktop/persp-analysis_A18-master_2/Assignments/A2"}\NormalTok{)}
\KeywordTok{library}\NormalTok{(dplyr)}
\KeywordTok{library}\NormalTok{(tidyverse)}

\CommentTok{#1. Imputing Age and Gender}
\NormalTok{best_income <-}\StringTok{ }\KeywordTok{read_delim}\NormalTok{(}\DataTypeTok{file =} \StringTok{'BestIncome.txt'}\NormalTok{, }\DataTypeTok{delim =} \StringTok{','}\NormalTok{,}\DataTypeTok{col_names =} \KeywordTok{c}\NormalTok{(}\StringTok{"lab_inc"}\NormalTok{, }\StringTok{"cap_inc"}\NormalTok{, }\StringTok{"hgt"}\NormalTok{, }\StringTok{"wgt"}\NormalTok{))}
\NormalTok{survey_income <-}\StringTok{ }\KeywordTok{read_delim}\NormalTok{(}\DataTypeTok{file =} \StringTok{'SurvIncome.txt'}\NormalTok{, }\DataTypeTok{delim =} \StringTok{','}\NormalTok{, }\DataTypeTok{col_names =} \KeywordTok{c}\NormalTok{(}\StringTok{"tot_inc"}\NormalTok{, }\StringTok{"wgt"}\NormalTok{, }\StringTok{"age"}\NormalTok{, }\StringTok{"female"}\NormalTok{))}
\KeywordTok{summary}\NormalTok{(best_income)}
\end{Highlighting}
\end{Shaded}

\begin{verbatim}
##     lab_inc         cap_inc           hgt             wgt       
##  Min.   :22918   Min.   : 1495   Min.   :58.18   Min.   :114.5  
##  1st Qu.:51624   1st Qu.: 8612   1st Qu.:63.65   1st Qu.:143.3  
##  Median :56969   Median : 9970   Median :65.00   Median :149.9  
##  Mean   :57053   Mean   : 9986   Mean   :65.01   Mean   :150.0  
##  3rd Qu.:62408   3rd Qu.:11340   3rd Qu.:66.36   3rd Qu.:156.7  
##  Max.   :90060   Max.   :19882   Max.   :72.80   Max.   :185.4
\end{verbatim}

\begin{Shaded}
\begin{Highlighting}[]
\KeywordTok{summary}\NormalTok{(survey_income)}
\end{Highlighting}
\end{Shaded}

\begin{verbatim}
##     tot_inc           wgt              age            female   
##  Min.   :31816   Min.   : 99.66   Min.   :25.74   Min.   :0.0  
##  1st Qu.:58350   1st Qu.:130.18   1st Qu.:41.03   1st Qu.:0.0  
##  Median :65281   Median :149.76   Median :44.96   Median :0.5  
##  Mean   :64871   Mean   :149.54   Mean   :44.84   Mean   :0.5  
##  3rd Qu.:71749   3rd Qu.:170.15   3rd Qu.:48.82   3rd Qu.:1.0  
##  Max.   :92556   Max.   :196.50   Max.   :66.53   Max.   :1.0
\end{verbatim}

\begin{Shaded}
\begin{Highlighting}[]
\CommentTok{#a) The scatterplot shows that the majority of females are 150 pounds or less, so we can impute gender based on whether or not the observation is 150 pounds or less. As for age, there is no clear trend, so we take the mean age and apply it to all observations. }

\KeywordTok{ggplot}\NormalTok{(}\DataTypeTok{data=}\NormalTok{survey_income) +}
\StringTok{  }\KeywordTok{geom_point}\NormalTok{(}\DataTypeTok{mapping =} \KeywordTok{aes}\NormalTok{(}\DataTypeTok{x=}\NormalTok{wgt, }\DataTypeTok{y=}\NormalTok{tot_inc, }\DataTypeTok{color =} \KeywordTok{as.factor}\NormalTok{(female)))}
\end{Highlighting}
\end{Shaded}

\includegraphics{MACS_PS2_files/figure-latex/unnamed-chunk-1-1.pdf}

\begin{Shaded}
\begin{Highlighting}[]
\KeywordTok{ggplot}\NormalTok{(}\DataTypeTok{data=}\NormalTok{survey_income) +}
\StringTok{  }\KeywordTok{geom_point}\NormalTok{(}\DataTypeTok{mapping =} \KeywordTok{aes}\NormalTok{(}\DataTypeTok{x=}\NormalTok{age, }\DataTypeTok{y=}\NormalTok{tot_inc, }\DataTypeTok{color =} \KeywordTok{as.factor}\NormalTok{(female)))}
\end{Highlighting}
\end{Shaded}

\includegraphics{MACS_PS2_files/figure-latex/unnamed-chunk-1-2.pdf}

\begin{Shaded}
\begin{Highlighting}[]
\CommentTok{#b)}
\NormalTok{best_income$gender <-}\StringTok{ }\KeywordTok{ifelse}\NormalTok{(best_income$wgt <}\StringTok{ }\DecValTok{150}\NormalTok{, }\DecValTok{1}\NormalTok{, }\DecValTok{0}\NormalTok{)}
\NormalTok{best_income$age <-}\StringTok{ }\KeywordTok{mean}\NormalTok{(survey_income$age)}

\CommentTok{#c)}
\KeywordTok{summary}\NormalTok{(best_income$gender)}
\end{Highlighting}
\end{Shaded}

\begin{verbatim}
##    Min. 1st Qu.  Median    Mean 3rd Qu.    Max. 
##  0.0000  0.0000  1.0000  0.5019  1.0000  1.0000
\end{verbatim}

\begin{Shaded}
\begin{Highlighting}[]
\KeywordTok{summary}\NormalTok{(best_income$age)}
\end{Highlighting}
\end{Shaded}

\begin{verbatim}
##    Min. 1st Qu.  Median    Mean 3rd Qu.    Max. 
##   44.84   44.84   44.84   44.84   44.84   44.84
\end{verbatim}

\begin{Shaded}
\begin{Highlighting}[]
\KeywordTok{sd}\NormalTok{(best_income$gender)}
\end{Highlighting}
\end{Shaded}

\begin{verbatim}
## [1] 0.5000214
\end{verbatim}

\begin{Shaded}
\begin{Highlighting}[]
\KeywordTok{sd}\NormalTok{(best_income$age)}
\end{Highlighting}
\end{Shaded}

\begin{verbatim}
## [1] 0
\end{verbatim}

\begin{Shaded}
\begin{Highlighting}[]
\CommentTok{#d)}
\NormalTok{correlation <-}\StringTok{ }\KeywordTok{cor}\NormalTok{(best_income)}
\KeywordTok{round}\NormalTok{(correlation, }\DecValTok{2}\NormalTok{)}
\end{Highlighting}
\end{Shaded}

\begin{verbatim}
##         lab_inc cap_inc   hgt   wgt gender age
## lab_inc    1.00    0.01  0.00  0.00  -0.01  NA
## cap_inc    0.01    1.00  0.02  0.01  -0.01  NA
## hgt        0.00    0.02  1.00  0.17  -0.14  NA
## wgt        0.00    0.01  0.17  1.00  -0.80  NA
## gender    -0.01   -0.01 -0.14 -0.80   1.00  NA
## age          NA      NA    NA    NA     NA   1
\end{verbatim}

\begin{Shaded}
\begin{Highlighting}[]
\CommentTok{#2. Stationarity and Data Drift}
\NormalTok{income_intel <-}\StringTok{ }\KeywordTok{read_delim}\NormalTok{(}\DataTypeTok{file =} \StringTok{'IncomeIntel.txt'}\NormalTok{, }\DataTypeTok{delim =} \StringTok{','}\NormalTok{,}\DataTypeTok{col_names =} \KeywordTok{c}\NormalTok{(}\StringTok{"grad_year"}\NormalTok{, }\StringTok{"gre_qnt"}\NormalTok{, }\StringTok{"salary_p4"}\NormalTok{))}

\CommentTok{#a)}
\NormalTok{lm_s_g =}\StringTok{ }\KeywordTok{lm}\NormalTok{(income_intel$salary_p4 ~}\StringTok{ }\NormalTok{income_intel$gre_qnt)}
\KeywordTok{summary}\NormalTok{(lm_s_g)}
\end{Highlighting}
\end{Shaded}

\begin{verbatim}
## 
## Call:
## lm(formula = income_intel$salary_p4 ~ income_intel$gre_qnt)
## 
## Residuals:
##    Min     1Q Median     3Q    Max 
## -28761  -7049   -293   6549  37666 
## 
## Coefficients:
##                       Estimate Std. Error t value Pr(>|t|)    
## (Intercept)          89541.293    878.764  101.89   <2e-16 ***
## income_intel$gre_qnt   -25.763      1.365  -18.88   <2e-16 ***
## ---
## Signif. codes:  0 '***' 0.001 '**' 0.01 '*' 0.05 '.' 0.1 ' ' 1
## 
## Residual standard error: 10460 on 998 degrees of freedom
## Multiple R-squared:  0.2631, Adjusted R-squared:  0.2623 
## F-statistic: 356.3 on 1 and 998 DF,  p-value: < 2.2e-16
\end{verbatim}

\begin{Shaded}
\begin{Highlighting}[]
\CommentTok{#b)}
\KeywordTok{ggplot}\NormalTok{(}\DataTypeTok{data=}\NormalTok{income_intel) +}
\StringTok{  }\KeywordTok{geom_jitter}\NormalTok{(}\DataTypeTok{mapping =} \KeywordTok{aes}\NormalTok{(}\DataTypeTok{x=}\NormalTok{grad_year, }\DataTypeTok{y=}\NormalTok{gre_qnt, }\DataTypeTok{color =} \KeywordTok{as.factor}\NormalTok{(grad_year), }\DataTypeTok{alpha=}\NormalTok{.}\DecValTok{05}\NormalTok{))}
\end{Highlighting}
\end{Shaded}

\includegraphics{MACS_PS2_files/figure-latex/unnamed-chunk-1-3.pdf}

\begin{Shaded}
\begin{Highlighting}[]
\CommentTok{#The problem with using this variable to test my hypothesis is that the GRE quant scoring scale changed in 2011. See below for the code that implements changing the scale for old scores. }

\NormalTok{income_intel$new_gre_qnt <-}\StringTok{ }\KeywordTok{with}\NormalTok{(income_intel, }\KeywordTok{ifelse}\NormalTok{(grad_year <}\StringTok{ }\DecValTok{2011}\NormalTok{, gre_qnt *}\StringTok{ }\DecValTok{170} \NormalTok{/}\StringTok{ }\DecValTok{800}\NormalTok{, gre_qnt))}

\CommentTok{#c)}
\KeywordTok{ggplot}\NormalTok{(}\DataTypeTok{data=}\NormalTok{income_intel) +}
\StringTok{  }\KeywordTok{geom_jitter}\NormalTok{(}\DataTypeTok{mapping =} \KeywordTok{aes}\NormalTok{(}\DataTypeTok{x=}\NormalTok{grad_year, }\DataTypeTok{y=}\NormalTok{salary_p4))}
\end{Highlighting}
\end{Shaded}

\includegraphics{MACS_PS2_files/figure-latex/unnamed-chunk-1-4.pdf}

\begin{Shaded}
\begin{Highlighting}[]
\CommentTok{# The problem is that inflation is not accounted for, since salaries have the same distribution but higher every year. I used Rick's solution to detrend the variable below. }

\NormalTok{by_grad_year <-}\StringTok{ }\KeywordTok{group_by}\NormalTok{(income_intel, grad_year)}
\NormalTok{avg_inc_by_year <-}\StringTok{ }\KeywordTok{summarise}\NormalTok{(by_grad_year, }\DataTypeTok{mean_salary=}\KeywordTok{mean}\NormalTok{(salary_p4))}

\NormalTok{avg_growth_rate <-}\StringTok{ }\KeywordTok{mean}\NormalTok{(}\KeywordTok{diff}\NormalTok{(avg_inc_by_year$mean_salary, }\DataTypeTok{lag =} \DecValTok{1}\NormalTok{, }\DataTypeTok{differences =} \DecValTok{1}\NormalTok{)/((}\KeywordTok{slice}\NormalTok{(avg_inc_by_year, }\DecValTok{1}\NormalTok{:}\DecValTok{12}\NormalTok{))$mean_salary))}
\NormalTok{avg_growth_rate}
\end{Highlighting}
\end{Shaded}

\begin{verbatim}
## [1] 0.03083535
\end{verbatim}

\begin{Shaded}
\begin{Highlighting}[]
\NormalTok{income_intel$adj_salary <-}\StringTok{ }\NormalTok{income_intel$salary_p4/((}\DecValTok{1}\NormalTok{+avg_growth_rate)**(income_intel$grad_year -}\StringTok{ }\DecValTok{2001}\NormalTok{))}

\KeywordTok{ggplot}\NormalTok{(}\DataTypeTok{data=}\NormalTok{income_intel) +}
\StringTok{  }\KeywordTok{geom_jitter}\NormalTok{(}\DataTypeTok{mapping =} \KeywordTok{aes}\NormalTok{(}\DataTypeTok{x=}\NormalTok{new_gre_qnt, }\DataTypeTok{y=}\NormalTok{adj_salary))}
\end{Highlighting}
\end{Shaded}

\includegraphics{MACS_PS2_files/figure-latex/unnamed-chunk-1-5.pdf}

\begin{Shaded}
\begin{Highlighting}[]
\CommentTok{#d)}
\NormalTok{new_lm_s_g =}\StringTok{ }\KeywordTok{lm}\NormalTok{(income_intel$adj_salary ~}\StringTok{ }\NormalTok{income_intel$new_gre_qnt)}
\KeywordTok{summary}\NormalTok{(new_lm_s_g)}
\end{Highlighting}
\end{Shaded}

\begin{verbatim}
## 
## Call:
## lm(formula = income_intel$adj_salary ~ income_intel$new_gre_qnt)
## 
## Residuals:
##      Min       1Q   Median       3Q      Max 
## -20213.6  -4783.4    123.4   4793.5  23219.5 
## 
## Coefficients:
##                          Estimate Std. Error t value Pr(>|t|)    
## (Intercept)              66834.37    6968.68   9.591   <2e-16 ***
## income_intel$new_gre_qnt   -34.97      44.99  -0.777    0.437    
## ---
## Signif. codes:  0 '***' 0.001 '**' 0.01 '*' 0.05 '.' 0.1 ' ' 1
## 
## Residual standard error: 7137 on 998 degrees of freedom
## Multiple R-squared:  0.0006052,  Adjusted R-squared:  -0.0003962 
## F-statistic: 0.6043 on 1 and 998 DF,  p-value: 0.4371
\end{verbatim}

\begin{Shaded}
\begin{Highlighting}[]
\CommentTok{#3. Assessment of Kossinets and Watts (2009)}
\CommentTok{#}
\CommentTok{#(a) State the research question of this paper. The research question is the}
\CommentTok{#undamental question that the paper is trying to answer. The research}
\CommentTok{#question should be one sentence long and should end with a question mark}
\CommentTok{#“?”. An example of a research question is, “What is the effect of an extra}
\CommentTok{#hour of storm advisory in a county on births nine months later in that}
\CommentTok{#county?”}
\CommentTok{#(b) Describe the data that the authors used. How many data sources? How}
\CommentTok{#many observations (this question could have multiple dimensions)? What}
\CommentTok{#time period did the data span? Where can you find a description and}
\CommentTok{#definition of all the variables?}
\CommentTok{#(c) Highlight a potential problem that the data cleaning process might introduce}
\CommentTok{#in a way that diminishes the authors’ ability to answer the research}
\CommentTok{#question.}
\CommentTok{#(d) In this paper, the underlying theoretical construct is “social relationships”}
\CommentTok{#and the data are e-mail logs linked to other characteristics of the senders}
\CommentTok{#and receivers. Discuss one weakness of this match of data source and}
\CommentTok{#theoretical construct and describe how the authors address this weakness.}
\end{Highlighting}
\end{Shaded}


\end{document}
